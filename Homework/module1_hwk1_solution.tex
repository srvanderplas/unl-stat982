\documentclass{article}
% Change "article" to "report" to get rid of page number on title page
\usepackage{amsmath,amsfonts,amsthm,amssymb}
\usepackage{setspace}
\usepackage{Tabbing}
\usepackage{fancyhdr}
\usepackage{lastpage}
\usepackage{extramarks}
\usepackage{chngpage}
\usepackage{soul,color}
\usepackage{graphicx,float,wrapfig}
% \usepackage{enumerate}
\usepackage[shortlabels]{enumitem}

% In case you need to adjust margins:
\topmargin=-0.45in      %
\evensidemargin=0in     %
\oddsidemargin=0in      %
\textwidth=6.5in        %
\textheight=9.0in       %
\headsep=0.25in         %
\headheight=15pt

% Homework Specific Information
\newcommand{\hmwkTitle}{Module 1, Homework 1}
\newcommand{\hmwkDueDate}{September 6, 2022}
\newcommand{\hmwkAuthorName}{}
\newcommand{\hmwkClass}{Stat 982}

% Setup the header and footer
\pagestyle{fancy}                                                       %
\lhead{\hmwkAuthorName}                                                 %
\chead{\hmwkClass\: \hmwkTitle}  %
\rhead{\firstxmark}                                                     %
\lfoot{\lastxmark}                                                      %
\cfoot{}                                                                %
\rfoot{Page\ \thepage\ of\ \pageref{LastPage}}                          %
\renewcommand\headrulewidth{0.4pt}                                      %
\renewcommand\footrulewidth{0.4pt}                                      %


%%%%%%%%%%%%%%%%%%%%%%%%%%%%%%%%%%%%%%%%%%%%%%%%%%%%%%%%%%%%%
% Some tools
\newcommand{\enterProblemHeader}[1]{\nobreak\extramarks{#1}{#1 continued on next
page\ldots}\nobreak\nobreak\extramarks{#1 (continued)}{#1
continued on next page\ldots}\nobreak}%

\newcommand{\exitProblemHeader}[1]{\nobreak\extramarks{#1 (continued)}{#1
continued on next page\ldots}\nobreak\nobreak\extramarks{#1}{}\nobreak}%

\newlength{\labelLength}
\newcommand{\labelAnswer}[2]
  {\settowidth{\labelLength}{#1}%
   \addtolength{\labelLength}{0.25in}%
   \changetext{}{-\labelLength}{}{}{}%
   \noindent\fbox{\begin{minipage}[c]{\columnwidth}#2\end{minipage}}%
   \marginpar{\fbox{#1}}%

   % We put the blank space above in order to make sure this
   % \marginpar gets correctly placed.
   \changetext{}{+\labelLength}{}{}{}}%

\setcounter{secnumdepth}{0}
\newcommand{\homeworkProblemName}{}%
\newcounter{homeworkProblemCounter}%
\newenvironment{homeworkProblem}[1][Problem \arabic{homeworkProblemCounter}]%
  {\stepcounter{homeworkProblemCounter}%
   \renewcommand{\homeworkProblemName}{#1}%
   \section{\homeworkProblemName}%
   \enterProblemHeader{\homeworkProblemName}}%
  {\exitProblemHeader{\homeworkProblemName}}%

\newcommand{\problemAnswer}[1]{\vspace{.1in}\noindent\fbox{\begin{minipage}[c]{.87\textwidth}#1\end{minipage}}\vspace{.1in}}%

\newcommand{\problemLAnswer}[1]
  {\labelAnswer{\homeworkProblemName}{#1}}

\newcommand{\homeworkSectionName}{}%
\newlength{\homeworkSectionLabelLength}{}%
\newenvironment{homeworkSection}[1]%
  {% We put this space here to make sure we're not connected to the above.
   % Otherwise the changetext can do funny things to the other margin
   \renewcommand{\homeworkSectionName}{#1}%
   \settowidth{\homeworkSectionLabelLength}{\homeworkSectionName}%
   \addtolength{\homeworkSectionLabelLength}{0.25in}%
   \changetext{}{-\homeworkSectionLabelLength}{}{}{}%
   \subsection{\homeworkSectionName}%
   \enterProblemHeader{\homeworkProblemName\ [\homeworkSectionName]}}%
  {\enterProblemHeader{\homeworkProblemName}%

   % We put the blank space above in order to make sure this margin
   % change doesn't happen too soon (otherwise \sectionAnswer's can
   % get ugly about their \marginpar placement.
   \changetext{}{+\homeworkSectionLabelLength}{}{}{}}%

\newcommand{\sectionAnswer}[1]
  {\noindent\fbox{\begin{minipage}[c]{\columnwidth}#1\end{minipage}}%
  
\enterProblemHeader{\homeworkProblemName}\exitProblemHeader{\homeworkProblemName}%
   \marginpar{\fbox{\homeworkSectionName}}
   }%

\newcommand{\N}{\mathbb{N}}
\newcommand{\R}{\mathbb{R}}
% Make title
\title{\textmd{\textbf{\hmwkClass:\ \hmwkTitle}}\\\small{Due\ on\ \hmwkDueDate}}
\author{\textbf{\hmwkAuthorName}}
\date{}
%%%%%%%%%%%%%%%%%%%%%%%%%%%%%%%%%%%%%

\begin{document}
\maketitle

\begin{homeworkProblem} % Stat 643 Hw 4 Problem 2
Suppose that $X = (X_1, X_2, ..., X_n)$ has independent components, where each $X_i$ is generated as follows. For independent random variables $W_i \sim N(\mu, 1)$ and $Z_i \sim Poisson(\mu)$, $X_i = W_i$ with probability $p$ and $X_i = Z_i$ with probability $1-p$. Suppose that $\mu \in [0,\infty)$. Use the factorization theorem and find low-dimensional sufficient statistics in the cases that:

\begin{enumerate}[(a)]
\item $p$ is known to be $\frac{1}{2}$
\item $p\in[0,1]$ is unknown
\end{enumerate}

Note: In the first case the parameter space is $\Theta=\left\{\frac{1}{2}\right\}\times[0,\infty)$, while in the second case it is $\Theta=[0,1]\times[0,\infty)$.

\problemAnswer{
Let $\lambda = m+v$, where $m$ is the Lebesgue measure on $\mathbb{R}^+=[0,\infty)$ and $v$ is the counting measure on $\mathbb{N}=\{0, 1, 2, ...\}$. Note that $X_1\overset{d}{=} R_1W_1 + (1-R_1)Z_1$, where $R_1$ is independent of $(W_1, Z_1)$ and $R_1\sim\text{Bernoulli}(P)$. For any $A\in\mathcal{B}^1(\mathbb{R}^1)$, we have \begin{align*}P^{X_1}(A) &= P(X_1\in A) \\
&= P(R_1W_1+(1-R_1)Z_1\in A)\\
&=P(R_1W_1 + (1-R_1)Z_1 \in A|R_1=1)\cdot P(R_1=1) + P(R_1W_1 + (1-R_1)Z_1\in A|R_1=0)P(R_1=0)\\
& = p \cdot P(W_1 \in A) + (1-p)\cdot P(Z_1\in A)\\
& = p\cdot P^{W_1}(A)+(1-p)P^{Z_1}(A)\\
& = \int_A p\frac{1}{\sqrt{2\pi}}\exp\left[-\frac{(x-\mu)^2}{2}\right]dm(x) + \int_A (1-p)\cdot\frac{e^-\mu\cdot\mu^x}{x!}dv(x)\\
& = \int_A p\frac{1}{\sqrt{2\pi}}\exp\left[-frac{(x-\mu)^2}{2}\right]\cdot I[x\notin \mathbb{N}]d\lambda(x) + \int_A (1-p)\frac{e^{-\mu}\mu^x}{x!}I\left[x\in\mathbb{N}\right]d\lambda(x)
\end{align*}
Thus, $\frac{dP^{X_1}}{d\lambda}(x)=\left\{p\cdot\frac{1}{\sqrt{2\pi}}\exp\left[-\frac{(x-\mu)^2}{2}\right]\right\}^{I[x\notin\mathbb{N}]}\cdot\left\{(1-p)\frac{e^{-\mu}\mu^x}{x!}\right\}^{I\left[x\in\mathbb{N}\right]}$ a.e. $(\lambda)$\\
Let $\delta_i=I[x_i\notin\N]$, $i=1, ..., n$. Then for $x=(x_1, ..., x_n)$, 
\begin{align*}
\frac{dP^x}{d\lambda}(x) &= \prod_{i=1}^n \frac{dP^{x_i}}{d\lambda}(x_i)\\
&=\prod_{i=1}^{n}\left\{p\frac{1}{\sqrt{2\pi}}\exp\left[-\frac{(x_i-\mu)^2}{2}\right]\right\}^{\delta_i}\cdot\left\{(1-p)\frac{e^-\mu \mu^x}{x!}\right\}^{1-\delta_i}\\
&=\exp\left\{\left[\mu-\frac{\mu^2}{2} + \log\left(\frac{p}{1-p}\right)\right]\sum_{i=1}^n\delta_i + \mu\sum_{i=1}^n \delta_ix_i\right\}\cdot\mu^{\sum_{i=1}^n (1-\delta_i)x_i}\\
&\phantom{=}\cdot(2\pi)^{-\frac{1}{2}\sum_{i=1}^n\delta_i}\cdot\exp\left[-\frac{1}{2}\sum_{i=1}^n x_i^2\delta_i\right]\cdot\prod_{i=1}^n (x_i!)^{\delta_i - 1}(1-p)^n e^{-n\mu}
\end{align*}
By the factorization theorem, a low dimensional sufficient statistic for both (a) and (b) is given by $$T(X) = \left(\sum_{i=1}^nI[X_i\notin\N], \sum_{i=1}^n I[X_i\notin\N]\cdot X_i, \sum_{i=1}^n (1-I[X_i\notin\N])\cdot X_i\right)$$
}
\end{homeworkProblem}

\begin{homeworkProblem} % Stat 643 Hw 4 Problem 3
Suppose that $X^\prime$ is exponential with mean $\lambda ^{-1}$ (i.e. it has density $f_\lambda(x)=\lambda \exp(-\lambda x)I[x\geq0]$ with respect to the Lebesgue measure on $\mathbb{R}^1$), but that one only observes $X=X^\prime I[X^\prime\geq1]$. (There is interval censoring below $X^\prime = 1$). 

\begin{enumerate}[(a)]
\item Consider the measure $\mu$ on $\mathcal X=\{0\}\bigcup[1,\infty)$ consisting of a point mass of 1 at 0 plus the Lebesgue measure on $[1,\infty)$. Give a formula for the R-N derivative of $P_\lambda^X$ with respect to $\mu$ on $\mathcal X$.

\item Suppose that $X_1, ..., X_n$ are iid with the distribution $P_\lambda^X$. Find a two-dimensional sufficient statistic for this problem and argue that it is indeed sufficient.

\item Argue carefully that your statisic from the previous part is minimal sufficient.

\end{enumerate}
\end{homeworkProblem}
\begin{homeworkProblem}%Stat 643 Hw 4 Problem 6
% (Shao Exercise 2.25)
Let $X$ be a sample from $P\in\mathcal{P}$ where $\mathcal P$ is a family of distributions on $(\mathbb{R}^k, \mathcal{B}^k)$. Show that if $T(X)$ is sufficient for $\mathcal P$ and  $T = \psi(S)$, where $\psi$ is measurable and $S(X)$ is another statistic, then $S(X)$ is sufficient for $\mathcal P$. (Hint: Consider first the special case that $\mathcal P$ is dominated by a $\sigma$-finite measure $\mu$ and then the general case).

\end{homeworkProblem}

\begin{homeworkProblem}%Stat 643 Hw 4 Problem 7
% (Shao Exercise 2.41) 
Suppose that $X_1 , X_2, ... , X_n$ are iid $P_\theta$ for $\theta = (\theta_1, \theta_2) \in (0, 1) \times \{1, 2\}$, where $P_{(\gamma,1)}$ is the Poisson$(\gamma)$ distribution and $P(\gamma,2)$ is the Bernoulli$(\gamma)$ distribution. Find a two-
dimensional minimal sufficient statistic for $\theta$ (and argue carefully for minimal sufficiency).
\end{homeworkProblem}

\begin{homeworkProblem} % Stat 643 Hw 4 Problem 9
% (Shao Exercise 2.57) 
Suppose that $X_1, ..., X_n$ are iid $N(\gamma, \gamma^2)$ for $\gamma\in\mathbb{R}^1$. Find a minimal sufficient statistic and show that it is not complete.
\end{homeworkProblem}
\end{document}