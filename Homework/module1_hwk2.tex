\documentclass{article}
% Change "article" to "report" to get rid of page number on title page
\usepackage{amsmath,amsfonts,amsthm,amssymb}
\usepackage{setspace}
\usepackage{Tabbing}
\usepackage{fancyhdr}
\usepackage{lastpage}
\usepackage{extramarks}
\usepackage{chngpage}
\usepackage{soul,color}
\usepackage{graphicx,float,wrapfig}
% \usepackage{enumerate}
\usepackage[shortlabels]{enumitem}

% In case you need to adjust margins:
\topmargin=-0.45in      %
\evensidemargin=0in     %
\oddsidemargin=0in      %
\textwidth=6.5in        %
\textheight=9.0in       %
\headsep=0.25in         %
\headheight=15pt

% Homework Specific Information
\newcommand{\hmwkTitle}{Module 1, Homework 2}
\newcommand{\hmwkDueDate}{September 27, 2022}
\newcommand{\hmwkAuthorName}{}
\newcommand{\hmwkClass}{Stat 982}

% Setup the header and footer
\pagestyle{fancy}                                                       %
\lhead{\hmwkAuthorName}                                                 %
\chead{\hmwkClass\: \hmwkTitle}  %
\rhead{\firstxmark}                                                     %
\lfoot{\lastxmark}                                                      %
\cfoot{}                                                                %
\rfoot{Page\ \thepage\ of\ \pageref{LastPage}}                          %
\renewcommand\headrulewidth{0.4pt}                                      %
\renewcommand\footrulewidth{0.4pt}                                      %


%%%%%%%%%%%%%%%%%%%%%%%%%%%%%%%%%%%%%%%%%%%%%%%%%%%%%%%%%%%%%
% Some tools
\newcommand{\enterProblemHeader}[1]{\nobreak\extramarks{#1}{#1 continued on next
page\ldots}\nobreak\nobreak\extramarks{#1 (continued)}{#1
continued on next page\ldots}\nobreak}%

\newcommand{\exitProblemHeader}[1]{\nobreak\extramarks{#1 (continued)}{#1
continued on next page\ldots}\nobreak\nobreak\extramarks{#1}{}\nobreak}%

\newlength{\labelLength}
\newcommand{\labelAnswer}[2]
  {\settowidth{\labelLength}{#1}%
   \addtolength{\labelLength}{0.25in}%
   \changetext{}{-\labelLength}{}{}{}%
   \noindent\fbox{\begin{minipage}[c]{\columnwidth}#2\end{minipage}}%
   \marginpar{\fbox{#1}}%

   % We put the blank space above in order to make sure this
   % \marginpar gets correctly placed.
   \changetext{}{+\labelLength}{}{}{}}%

\setcounter{secnumdepth}{0}
\newcommand{\homeworkProblemName}{}%
\newcounter{homeworkProblemCounter}%
\newenvironment{homeworkProblem}[1][Problem \arabic{homeworkProblemCounter}]%
  {\stepcounter{homeworkProblemCounter}%
   \renewcommand{\homeworkProblemName}{#1}%
   \section{\homeworkProblemName}%
   \enterProblemHeader{\homeworkProblemName}}%
  {\exitProblemHeader{\homeworkProblemName}}%

\newcommand{\problemAnswer}[1]{\vspace{.1in}\noindent\fbox{\begin{minipage}[c]{.87\textwidth}#1\end{minipage}}\vspace{.1in}}%

\newcommand{\problemLAnswer}[1]
  {\labelAnswer{\homeworkProblemName}{#1}}

\newcommand{\homeworkSectionName}{}%
\newlength{\homeworkSectionLabelLength}{}%
\newenvironment{homeworkSection}[1]%
  {% We put this space here to make sure we're not connected to the above.
   % Otherwise the changetext can do funny things to the other margin
   \renewcommand{\homeworkSectionName}{#1}%
   \settowidth{\homeworkSectionLabelLength}{\homeworkSectionName}%
   \addtolength{\homeworkSectionLabelLength}{0.25in}%
   \changetext{}{-\homeworkSectionLabelLength}{}{}{}%
   \subsection{\homeworkSectionName}%
   \enterProblemHeader{\homeworkProblemName\ [\homeworkSectionName]}}%
  {\enterProblemHeader{\homeworkProblemName}%

   % We put the blank space above in order to make sure this margin
   % change doesn't happen too soon (otherwise \sectionAnswer's can
   % get ugly about their \marginpar placement.
   \changetext{}{+\homeworkSectionLabelLength}{}{}{}}%

\newcommand{\sectionAnswer}[1]
  {\noindent\fbox{\begin{minipage}[c]{\columnwidth}#1\end{minipage}}%
  
\enterProblemHeader{\homeworkProblemName}\exitProblemHeader{\homeworkProblemName}%
   \marginpar{\fbox{\homeworkSectionName}}
   }%

% Make title
\title{\textmd{\textbf{\hmwkClass:\ \hmwkTitle}}\\\small{Due\ on\ \hmwkDueDate}}
\author{\textbf{\hmwkAuthorName}}
\date{}
%%%%%%%%%%%%%%%%%%%%%%%%%%%%%%%%%%%%%

\begin{document}
\maketitle

\begin{homeworkProblem} % Stat 643 Hw 5 Problem 2
%(Shao Exercise 2.14) 
(Truncated exponential families) Suppose that distributions $P_\eta$ have R-N derivatives with respect to a $\sigma$-finite measure $\mu$ of the form $f_\eta(x) = K(\eta) \exp\left(\sum_{i=1}^k\eta_iT_i(x)\right)h(x)$. For a measurable set $A$ with $P_\eta(A)>0$, consider the family of distributions $Q_\eta^A$ on $A$ with R-N derivatives with respect to $\mu$ $$g_\eta(x)\propto f_\eta(x)I[x\in A]$$

Argue that this is an exponential family and say what you can about the natural parameter space for this family in comparison to that of the $P_\eta$ family.
\end{homeworkProblem}

\begin{homeworkProblem} % Stat 643 Hw 4 Problem 8
Suppose that $(X_1, Y_1), (X_2, Y_2), ..., (X_n, Y_n)$ are iid random vectors, and that $X_i$ and $Y_i$ are independently distributed as $N(\mu,\sigma^2_1)$ and $N(\mu,\sigma_2^2)$, respectively, with $\mathbf{\theta}=(\mu,\sigma_1^2,\sigma_2^2)\in\mathbb{R}^1\times(0,\infty)\times(0,\infty)$. Let $\overline{X}$ and $S^2_X$ be the sample mean and variance for the $X_i$'s and $\overline Y$ and $S_Y^2$ be the sample mean and sample variance for the $Y_i$'s. Show that $(\overline{X}, \overline{Y}, S^2_X, S^2_Y)$ is minimal sufficient but not boundedly complete. What is a first-order ancillary statistic here?
\end{homeworkProblem}

\begin{homeworkProblem}% Stat 643 Hw 5 Problem 4
% (Shao Exercise 2.45) 
Suppose that $X_1, X_2 , ..., X_n$ are iid $P_\theta$ for $\theta\in \mathbb{R}^1$ , where if $\theta\neq 0$ the distribution
$P_\theta$ is $N (\theta, 1)$, while $P_0$ is $N (\theta, 2)$. Show that $\overline{X} = \frac{1}{n}\sum_{i=1}^n$ is complete but not sufficient for $\theta$.
\end{homeworkProblem}

\begin{homeworkProblem}
Go through each line of the proof of the Factorization theorem and it's correlary and explain each step in a way that makes it clear that you understand it.
\end{homeworkProblem}

% \begin{homeworkProblem}% Theory of Point Estimation (1983) Problem 1.4.25
% If Y is distributed as $\Gamma (a, b)$, determine the distribution of $c\log Y$ and show that for fixed $a$ and varying $b$ it defines an exponential family.
% \end{homeworkProblem}
% 
% \begin{homeworkProblem} %Theory of Point Estimation (1983) Problem 1.5.10
% Show that the order statistics are minimal sufficient for the location family when $f$ is the density of 
% \begin{enumerate}[(a)]
% \item the double exponential distribution $D(0,1)$
% \item the Cauchy distribution $C(0,1)$
% \end{enumerate}
% \end{homeworkProblem}
\end{document}

Go over the two proofs (factorization theorem & correlary) and go through each step and explain it to us so that we know you understand.